%%%%%%%%% NEW SECTION: FAST SCAN %%%%%%%%%

\section{Possible Improvements}
\begin{itemize}
\item inhoudelijk: fast scan. wat heb ik tot nu toe gedaan, en waar wil ik heen. Was toch wel lastig.
\item in de code: in het gui-gedeelte: het plotten uitdrukken in de plotmethodes die ik gemaakt heb, ipv dat stukje code `opnieuw' uit te werken.
\item in de code: tests toevoegen om alle mogelijke manieren van afronden (bijv) te kunnen controleren en dus zeker weten dat dat stukje code doet wat je wilt dat het doet.
\item check in de code of de plaatjes eenheden op de assen meekrijgen. anders is dat een fantastische improvement. (check dat doet het)
\item nog meer de GUI loskoppelen van het data verwerken, vooral in het opslaangedeelte. Dat had een aparte klasse moeten worden.
\end{itemize}

%14:03 <marten> of alleen al iets wat dan specifiek "acceptatietest" heet: sluit 
%               alles aan, op een bepaalde (simpelste) configuratie (bijv niet 
%               vacuum), en draai script die je vertelt of het apparaat lijkt te 
%               reageren zoals je dan zou verwachten (binnen bepaalde marges)
%14:04 <marten> dus iets wat niet randgevallen checkt, maar juist kijkt of alles 
%               in het simpelste geval nog werkt
%14:04 <marten> dat kun je dan ook gebruiken om te kijken of er geen motoren 
%               stuk zijn gegaan na een jaar op de plank
%14:04 <marten> en of je alles correct hebt aangesloten
%14:05 <marten> met het idee dat als die test slaagt, dat je dan m kunt gaan 
%               gebruiken voor je specifiek gewenste setup
%14:05 <marten> en met redelijke zekerheid kunt zeggen dat je logische 
%               resultaten zult krijgen


