%%%%%%%%% NEW SECTION: STEP-INTEGRATE PROGRAM %%%%%%%%%

\section{Step-Integrate Program}

%Geef hier een blokdiagram van je programma, waarin aangegeven wordt hoe de data wordt ingevoerd, hoe de scan wordt uitgevoerd en hoe de data storage en processing (FFT, WIndowing, complex FFT?) wordt gedaan. Ter verduidelijking van de processing zou je een dummy interferogram (b.v. een top-hat) door je data processing kunnen sturen , om aan te geven wat het effect van de windowing is (en evt. het effect als je niet een correcte zero path hebt)

The code consists of a few different classes that talk to the hardware, or do measurements with the returned parameters. The classes \verb!C876_160!, \verb!SR510!, and \verb!EthernetAdapter! do the interaction with the hardware, with the commands as dictated in the corresponding manuals. The classes \verb!Parameters!, \verb!Plotting!, and \verb!Fourier! interact with the data that comes from the instruments. \verb!ReadoutSession! does the actual measurements, and a \verb!NepSession! has been made to test the rest of the code without having to use the actual hardware. The code comes together in the class \verb!Frame1!, where the different actions are coupled to the buttons in the GUI. This class also makes use of the file \verb!gui_stepintegrate.py!, which is the code for the GUI as generated by wxFormbuilder. See also figure~\ref{fig:blokdiagram}.


\begin{figure}[h!tb]
 \begin{center}
  \includegraphics[width=\textwidth]{figures/blokdiagram_classes.pdf}
  \caption{The classes in use in the program. The class `NepSession' is for testing purposes only.}
  \label{fig:blokdiagram}
 \end{center}
\end{figure}

\subsection{Addresses}
A few addresses have been hardcoded into the software. The stage controller is hardcoded to be on the (virtual) COM 3 port. The EthernetAdapter should have host '192.168.11.201' and port 1234, with the ethernet card in the computer the address '192.168.11.200'. Finally the lock in amplifier has to have GPIB address 8, because the EthernetAdapter is coded that way.

